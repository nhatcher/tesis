\newpage
\begin{center}
\textbf{RESUMEN}
\end{center}
En este trabajo discutimos diversos problemas relacionados con la structura y geometr�a del superespacio, haciendo un an�lisis lo m�s minucioso posible de las ideas que se han propuesto hasta ahora. Proponemos a su vez nuevas formas de tratar el superespacio, especialmente las teorias de super Maxwell. Hacemos tambien una breve introducci�n al superespacio arm�nico y a la manera en la que las superpart�culas se acoplan a los campos geom�tricos de super Maxwell. en la parte final del trabajo buscamos la mec�nica cu�ntica de las superpart�culas masivas en diversas dimensiones. Estudiamos tanto part�culas masivas normales como part�culas que vivan en un espacio con cargas centrales que pueden proceder de teor�as extensas en dimensiones m�s altas. Dilucidamos una nueva manera de realuizar la mec�nca cu�ntica con ligaduras de segunda clase basado en un potente m�todo de operadores de proyecci�n que nos permite realizar �lgebras no commmutativas de operadores de posici�n.
\newline\vskip 2cm

\begin{center}
Palabras claves: \textbf{superpart�cula, super Maxwell, cuantizaci�n covariante, superespacio}
\end{center}
